\documentclass[12pt]{article}
\usepackage[utf8]{inputenc}
\usepackage[margin=1.5in]{geometry}
\usepackage{natbib}
\bibliographystyle{apalike}

\usepackage{setspace}
\setstretch{1.2}

\title{Effects of Recruitment Methods on Survey Responses}
\author{Kevin Bi}
\date{\today}

\begin{document}

\maketitle

\section{Introduction}

This paper explores how recruitment methods appealing to altruism versus monetary self-interest affect responses to psychological surveys. Specifically, I examine differences along three dimensions: 
\begin{itemize}
    \item Participation rate - the proportion of people that complete the survey 
    \item Effort - how much work participants put into completing the survey 
    \item Responses - how responses to the survey differ between treatment groups
\end{itemize}

\subsection{Theoretical Bases}
\subsubsection{Participation Rate}
The exploration of participation rate is based on cognitive dissonance theory, and more specifically motivated reasoning. Previous research has found that people want to construct a model of the world where they can perceive themselves as good \citep{batson}. I test the strength of motivated reasoning versus monetary incentive by examining how willing people are to complete a short survey in response to altruistic appeals. Completing a short survey for no monetary incentive allows individuals to construct a cognitive model of themselves as morally good individuals. By comparing participation rates in response to altruistic appeals and monetary appeals, I am able to gauge the strength of the motivated reasoning effect. 

\subsubsection{Effort}
The exploration of survey effort is based on effort justification theory. \textbf{INSERT ARONSON CITATION} found that when people had to put effort into a task, they enjoyed it more. I test whether the decision to put effort into a survey without monetary incentive inspires further effort on the survey. 

\subsubsection{Responses}

\section{Methodology} 

\section{Results}

\section{Discussion}

\medskip
\bibliography{references}

\end{document}
